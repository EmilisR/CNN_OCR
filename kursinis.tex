\documentclass{VUMIFPSkursinis}
\usepackage{algorithmicx}
\usepackage{algorithm}
\usepackage{algpseudocode}
\usepackage{amsfonts}
\usepackage{amsmath}
\usepackage{bm}
\usepackage{caption}
\usepackage{color}
\usepackage{float}
\usepackage{graphicx}
\usepackage{listings}
\usepackage{subfig}
\usepackage{wrapfig}

% Titulinio aprašas
\university{Vilniaus universitetas}
\faculty{Matematikos ir informatikos fakultetas}
\department{Programų sistemų katedra}
\papertype{Kursinis darbas}
\title{Automobilių numerių atpažinimas naudojant neuroninius tinklus ir OCR}
\titleineng{Car Number Plate Recognition Using Neural Networks and OCR}
\status{3 kurso 5 grupės studentas}
\author{Emilis Ruzveltas}
\supervisor{dr. Vytautas Valaitis}
\date{Vilnius – \the\year}

% Nustatymai
\setmainfont{Palemonas}   % Pakeisti teksto šriftą į Palemonas (turi būti įdiegtas sistemoje)
\bibliography{bibliografija}

\begin{document}
\maketitle

\tableofcontents

\sectionnonum{Įvadas}
Diena iš dienos transporto priemonių numerių atpažinimo sistemos tampa vis aktualesnės šių dienų 
saugumo reikalavimuose. Didžioji dalis pastatų ir aikštelių yra aprūpintos apsaugos kameromis bei
visą parą vaizdą stebinčiais apsaugos darbuotojais. Esminė savybė norint užtikrinti išmanų vaizdo
stebėjimą ir apdorojimą - atpažinimo sistemų automatizavimas. Automobilių numerių atpažinimo sistemos
remiasi tuo, kad kiekviena transporto priemonė turi unikalų identifikacinį kodą, kuris leidžia
vienareikšmiškai nustatyti transporto priemonės savininką. Techniškai automobilių numerių atpažinimas
yra paveikslėlių apdorojimo programa, naudojantis specialiu algoritmu išgauti rezultatus iš nuotraukos.
Automatinis paveikslėlių atpažinimas turi platų spektrą pritaikimo sričių, tokių kaip nusikaltėlių sekimas,
bendruomenės aplinkos stebėjimas, automobilių patikra, automatinis kelių mokesčių surinkimas, išmanus eismo
reguliavimas.
\cite{bhushan2013license}

\sectionnonumnocontent{Darbo tikslas}
Sukurti programą, kuri gebėtų atpažinti lietuviškus automobilio numerius paveikslėlyje panaudojant
konvoliucinį neuroninį tinklą 
(toliau CNN\footnote{CNN - trumpinys angl. Convolutional Neural Network - konvoliucinis neuroninis tinklas}) (angl. Convolutional Neural Network)
ieškant numerio rėmelio bei Optinę ženklų atpažinimo 
(toliau OCR\footnote{OCR - trumpinys angl. Optical Character Recognition - Optinis ženklų atpažinimas.}) (angl. Optical Character Recognition) 
programą atpažinti numerį.

\sectionnonumnocontent{Uždaviniai}
Šiame darbe išsikeltam tikslui įgyvendinti apsibrėžti 5 uždaviniai:

\begin{itemize}
  \item Susigeneruoti didelį kiekį paveikslėlių.
  \item Apmokyti konvoliucinį neuroninį tinklą pateikiant sugeneruotus paveiklėlius.
  \item Panaudoti konvoliucinį neuroninį tinklą atpažįstant numerio rėmelį paveikslėlyje.
  \item Gautą numerio rėmelį pasiųsti į OCR programą ir gauti atpažintą numerį.
  \item Atvaizduoti gautus rezultatus pradiniame paveikslėlyje.
\end{itemize}

\section{Mokymo duomenų generavimas}
Norint sukurti realiai veikiančią programą, kuri naudotų neuroninį tinklą išgauti tikėtinam rezultatui,
tinklą reikia apmokyti su dideliu kiekiu duomenų. Apmokant bet kokį neuroninį tinklą turi būti būti pateiktas
duomenų rinkinys su norimu gauti rezultatu. 

\subsection{Generavimo parametrai}

Šiam tyrimui buvo pasirinkta generuoti duomenų rinkinį, kurių kiekvienas paveikslėlis būtų 128 pikselių
pločio ir 64 pikselių ilgio. Toks pasirinktas būdas užtikrina, kad neuroninis tinklas bus pajėgus suprasti 
paveikslėlio turinį, o dydis pakankamai mažas, kad būtų galima turėti efektyviai veikiantį neuroninį tinklą.

Sugeneruotų paveikslėlių pavyzdžiai:

\begin{minipage}{\linewidth}
  \centering
  \includegraphics[width=4cm]{train_images/1_good.png}
  \captionof{figure}{Tikimasis rezultatas \textbf{MNI144 1}.}
\end{minipage}

\begin{minipage}{\linewidth}
  \centering
  \includegraphics[width=4cm]{train_images/1_good2.png}
  \captionof{figure}{Tikimasis rezultatas \textbf{ZVM557 1}.}
\end{minipage}

\begin{minipage}{\linewidth}
  \centering
  \includegraphics[width=4cm]{train_images/0_small.png}
  \captionof{figure}{Tikimasis rezultatas \textbf{COH150 0} (per mažas numeris).}
\end{minipage}

\begin{minipage}{\linewidth}
  \centering
  \includegraphics[width=4cm]{train_images/0_truncated.png}
  \captionof{figure}{Tikimasis rezultatas \textbf{GTK311 0} (ne pilnas numeris).}
\end{minipage}

\begin{minipage}{\linewidth}
  \centering
  \includegraphics[width=4cm]{train_images/0_truncated2.png}
  \captionof{figure}{Tikimasis rezultatas \textbf{ABN046 0} (ne pilnas numeris).}
\end{minipage}

\begin{minipage}{\linewidth}
  \centering
  \includegraphics[width=4cm]{train_images/0_not_present.png}
  \captionof{figure}{Tikimasis rezultatas \textbf{KLF155 0} (nėra numerio).}
\end{minipage}

Pirmoji paveikslėlio rezultato dalis nurodo, koks yra teisingas numeris. Antroji - numerio rėmelio egzistavimas paveikslėlyje.
Jei reikšmė 1 - numeris yra tinkamas nuskaitymui, 0 - neatitinka kriterijų. 

Kriterijai, kurie nusako ar numerio rėmelis yra tinkamas apdorojimui:

\begin{itemize}
  \item Visas rėmelio plotas yra paveikslėlyje.
  \item Rėmelio plotis yra mažesnis nei 80\% paveikslėlio pločio.
  \item Rėmėlio aukštis yra mažesnis nei 87,5\% paveikslėlio aukščio.
  \item Rėmelio plotis yra didesnis nei 60\% paveikslėlio pločio.
  \item Rėmėlio aukštis yra didesnis nei 60\% paveikslėlio aukščio.
\end{itemize}

Su tokiais parametrais galima naudoti judantį 128x64 pikselių langelį, kuris judėtų po 8 pikselius ir kas kartą padidintų rėmelį
\begin{math}
  \sqrt{2}
\end{math}
kartų. Tokiu būdu užtikrinama, kad nebus praleista nei viena paveikslėlio vieta, o taip pat pakankamai efektyviai ir greitai pereinamas
visas paveikslėlis.

\begin{minipage}{\linewidth}
  \centering
  \includegraphics[width=10cm]{train_images/generating.png}
  \captionof{figure}{Paveikslėlių generavimo schema}
\end{minipage}

\subsubsection{Numerio generavimas}
Numeris ir rėmelio spalva generuojama atsitiktinai, tačiau tekstas turi būti tamsesnis negu rėmelis. Tokiu būdu bandoma atkurti realaus pasaulio apšvietimo variacijas.
Numeris generuojamas pagal Lietuvos Respublikos Valstybinių numerių formatą, kuris yra - 3 lotyniško alfabeto
raidės (išskyrus lietuvių kalboje nenaudojamas raides) ir 3 arabiški skaičiai. Generuojant numerį, 
atsitiktine tvarka parenkamos trys raidės iš 23 raidžių žodyno \textit{ABCDEFGHIJKLMNOPRSTUVYZ} bei 3 skaičiai 
iš skaičių žodyno \textit{0123456789}. Maksimalus galimas unikalių numerių skaičius siekia:
\begin{equation}
  23^3 * 10^3 = 12167000;
\end{equation}

\subsubsection{Transformacija}

Norint, kad tinklas efektyviai mokytųsi ir atpažintų nuotraukas realaus pasaulio sąlygomis, generuojant duomenų
rinkinį buvo pritaikyta remėlio transformacija. Tai atlikti buvo pasitelktas metodas generuoti atsitiktines
reikšmes X, Y, Z ašims ir pritaikyti Oilerio kampų metodą. \cite{slabaugh1999computing}
For example, yaw is allowed to vary a lot more than roll (you’re more likely to see a car turning a corner, than on its side).
Reikšmių rėžiai pasirinkti tokie, kuriuos labiausiai tikėtina sutikti realiame pasaulyje.
TODO

Ašių atsitiktinių reikšmių rėžiai:
\begin{equation}
  -0.3 <= X <= 0.3
\end{equation}
\begin{equation}
  -0.2 <= Y <= 0.2
\end{equation}
\begin{equation}
  -1.2 <= Z <= 1.2
\end{equation}

$TODO (paaiskinti euler_to_mat koda)$

\subsubsection{Kompozicija}
Turėti realų foną svarbu, kadangi tinklas turi išmokti surasti rėmelio kampus „nesukčiaudamas“. 
Naudojant juodą foną, tinklas gali daryti prielaidą, kad remėlis yra ten, kur nėra juodos spalvos, o tai būtų netikslu realiame pasaulyje.
Transformuotas automobilio numerio rėmelis sukomponuojamas su atsitiktine nuotrauka atsitiktinėje vietoje.
Atsitiktinių nuotraukų šaltiniu naudojamas daugiau nei 100,000 nuotraukų duomenų rinkinys \cite{xiao2010sun}.
Labai svarbu didelis kiekis nuotraukų, taip sumažinant riziką, kad neuroninis tinklas atsimins kiekvieną nuotrauką.

\begin{equation}
  A = 0.6; 
\end{equation}
\begin{equation}
  B = 0.875;
\end{equation}
\begin{equation}
  C = 1.5.
\end{equation}
kur:
\begin{itemize}
  \item A - minimalus numerio rėmelio plotis,
  \item B - maksimalus numerio rėmelio aukštis,
  \item C - dydžio variacijos koeficientas.
\end{itemize}
\begin{equation}
  min = (A + B) * 0.5 - (B - A) * 0.5 * C;
\end{equation}
\begin{equation}
  min = ((0.6 + 0.875) * 0.5) - ((0.875 - 0.6) * 0.5 * 1.5);
\end{equation}
\begin{equation}
  min = (1.475 * 0.5) - (0.275 * 0.5 * 1.5);
\end{equation}
\begin{equation}
  min = 0.7375 - 0.20625;
\end{equation}
\begin{equation}
  \textbf{min = 0.53125};
\end{equation}
\begin{equation}
  max = (A + B) * 0.5 + (B - A) * 0.5 * C;
\end{equation}
\begin{equation}
  max = ((0.6 + 0.875) * 0.5) + ((0.875 - 0.6) * 0.5 * 1.5);
\end{equation}
\begin{equation}
  max = (1.475 * 0.5) + (0.275 * 0.5 * 1.5);
\end{equation}
\begin{equation}
  max = 0.7375 + 0.20625;
\end{equation}
\begin{equation}
  \textbf{max = 0.94375};
\end{equation}
\begin{equation}
  x = R[min, max];
\end{equation}
\begin{equation}
  p = 1, \text{kai } x \in [A, B];
\end{equation}
\begin{equation}
  p = 0, \text{kai } x \in [A, B];
\end{equation}
kur:
\begin{itemize}
  \item min - minimalus generuojamas dydžio koeficientas,
  \item max - maksimalus generuojamas dydžio koeficientas,
  \item R - atsitiktinio skaičiaus generavimo funkcija tarp dviejų reikšmių,
  \item x - numerio rėmelio dydžio koeficientas lyginant su pradiniu dydžiu,
  \item p - jei 1 - rėmelis tinkamai egzistuoja nuotraukoje, 0 - rėmelis neegzistuoja arba yra netinkamas.
\end{itemize}



Atsitiktinių reikšmių rėžis, kuris nusako kurioje vietoje turėtų atsidurti rėmelis
\subsubsection{Triukšmo pridėjimas}
Triukšmas paveikslėlyje reikalingas, kadangi realiame pasaulyje pasitaiko, kad kameros sensorius generuoja triukšmus, o taip pat,
kad neuroninis tinklas nepersimokytų ir neskirstytų paveikslėlių pagal vieną konkrečią spalvą ar būtų priklausomas nuo „aštrių“ kampų.
Triukšmas paveikslėliui pridedamas pritaikant Gauso normalųjį skirstinį su reikšme 0.05.



\section{Neuroninio tinklo konfigūracija}

\section{Neuroninio tinklo apmokymas}

\section{Numerio rėmelio koordinačių gavimas naudojantis CNN}

\section{Numerio atpažinimas pasitelkiant OCR}

\section{Gautų rezultatų atvaizdavimas nuotraukoje}

\sectionnonum{Rezultatai ir išvados}
Rezultatų ir išvadų dalyje turi būti aiškiai išdėstomi pagrindiniai darbo
rezultatai (kažkas išanalizuota, kažkas sukurta, kažkas įdiegta) ir pateikiamos
išvados (daromi nagrinėtų problemų sprendimo metodų palyginimai, teikiamos
rekomendacijos, akcentuojamos naujovės).

Kas atlikta:

1. Sugeneruota daug paveiksliukų, skirtų apmokyti neuroninį tinklą.

2. Panaudotas Google Street View konvoliucinis neuroninis tinklas.

3. Tinklas apmokytas su 100.000 sugeneruotų paveikslėlių.

4. Tinklas ištestuotas su tikromis nuotraukomis, kuriose yra lietuviški automobilių numeriai.

Išvados:

1. Pavyko atpažinti automobilių numerius naudojantis konvoliuciniu neuroniniu tinklu.

2. Sukurtas neuroninis tinklas nėra pakankamai spartus, norint jį naudoti realiu laiku.

3. Tinklas atpažįsta tik pakankamai horizontaliai tiesiai padarytas nuotraukas.


\printbibliography[heading=bibintoc]  % Šaltinių sąraše nurodoma panaudota
% literatūra, kitokie šaltiniai. Abėcėlės tvarka išdėstomi darbe panaudotų
% (cituotų, perfrazuotų ar bent paminėtų) mokslo leidinių, kitokių publikacijų
% bibliografiniai aprašai.  Šaltinių sąrašas spausdinamas iš naujo puslapio.
% Aprašai pateikiami netransliteruoti. Šaltinių sąraše negali būti tokių
% šaltinių, kurie nebuvo paminėti tekste.

% \sectionnonum{Sąvokų apibrėžimai}
\sectionnonum{Santrumpos}
Sąvokų apibrėžimai ir santrumpų sąrašas sudaromas tada, kai darbo tekste
vartojami specialūs paaiškinimo reikalaujantys terminai ir rečiau sutinkamos
santrumpos.

\appendix  % Priedai
% Prieduose gali būti pateikiama pagalbinė, ypač darbo autoriaus savarankiškai
% parengta, medžiaga. Savarankiški priedai gali būti pateikiami ir
% kompaktiniame diske. Priedai taip pat numeruojami ir vadinami. Darbo tekstas
% su priedais susiejamas nuorodomis.

\end{document}
